\documentclass[11pt,a4paper]{article}
\usepackage[utf8]{inputenc}
\usepackage[spanish]{babel}
\usepackage{amsmath,amssymb,amsthm,amsfonts}
\usepackage[paperheight=27.5cm, paperwidth=21cm,% Set the height and width of the paper
includehead,
nomarginpar,% We don't want any margin paragraphs
textwidth=17cm,% Set \textwidth to 18cm
textheight=18.4cm,% Set \textwidth to 22cm
headheight=46mm,% Set \headheight to 10mm
]{geometry}

% Useful packages
\usepackage[colorlinks=true, allcolors=blue]{hyperref}
\usepackage{indentfirst}
\usepackage{latexsym}
\usepackage{amsmath,amssymb}
\usepackage{enumitem}
\usepackage{float}
\usepackage{soul}
\usepackage{multirow}
\usepackage{makecell}
\usepackage{amssymb}
\usepackage[table]{xcolor}
\usepackage{graphicx}
\usepackage{caption}
\usepackage{xspace}
\usepackage{amsmath}
\usepackage{color}
\usepackage{tikz}
\usetikzlibrary{fit,automata,arrows}
\usetikzlibrary{patterns,hobby}


\newcommand{\X}{\cellcolor{gray!55} \checkmark}
\newcommand{\N}{\mathbb{N}}
\newcommand{\R}{\mathbb{R}}
\newcommand{\Z}{\mathbb{Z}}
\newcommand{\Q}{\mathbb{Q}}
\newcommand{\C}{\mathbb{C}}
\newcommand{\fin}{\hfill\vbox{\hrule height 5pt width 5pt }\bigskip}

\newcommand{\NP}{{\sf NP}}

\usepackage{helvet}
\renewcommand{\familydefault}{\sfdefault}

\usepackage{fancyhdr}

\fancypagestyle{plain}{%
  \fancyhf{}%
  \fancyfoot[R]{\thepage}%
  \renewcommand{\headrulewidth}{0.4pt}% Line at the header invisible
  \renewcommand{\footrulewidth}{0.02pt}% Footer line not visible with 0pt

\fancyhead[C]{
\begin{center}
{\Large
\textbf{UNIVERSIDAD DE BUENOS AIRES \\
Facultad de Ciencias Exactas y Naturales}}\\

\bigskip 

\large \textbf{
Investigación Operativa 1C 2025\\
Proyecto de Programación Lineal Entera }
\end{center}}
}

\usepackage{tcolorbox}
\newtcolorbox{mybox}{colback=gray!30,
boxrule=1pt,arc=0pt,boxsep=0pt,left=2pt,right=2pt,leftrule=1pt}

\renewcommand\spanishtablename{Tabla}


\begin{document}

%\title{.....}
%\author{Felipe Luc Pasquet}
%\date{\today}

%\maketitle

\pagestyle{plain}

%\begin{mybox}
%\centering \textbf{PLAN DE INVESTIGACIÓN}
%\end{mybox}

\medskip

\textbf{Nombre del alumno/s:} Felipe Pasquet, Juan I Catania \\


\textbf{Nombre del profesor/es:} Isabel Méndez-Díaz, Paula Zabala \\

\textbf{Disciplina/área del proyecto:} Investigación Operativa - Programación Lineal Entera

\section*{Introducción} {\small Queremos encontrar el recorrido óptimo para una empresa de distribución de productos que está buscando reducir costos al contratar bicicletas para que hagan algunos envíos a distancias cortas desde donde pasa el camión.}\

%.... \cite{Uriel-tesis,G-J-S-NPC}

\section*{Objetivo} {\small En este trabajo práctico nos interesa poder comparar los costos entre continuar con la metodología actual y la nueva manera de distribución. Para esto vamos a contar con la siguiente información:\\
\begin{enumerate}
    \item La cantidad de clientes a quienes se debe satisfacer la demanda, $cant\_clientes$.
\item Hay productos que exigen refrigeración, $r_i \in \{0,1\}$.
\item La distancia $d_{ij}$ entre todo par de clientes i y j y el costo $c_{ij}$ de desplazar el camión desde i a j.
\item Cada repartidor a bici que se contrate tiene un costo de $costo\_repartidor$ por cada cliente que deba visitar.
\item La distancia máxima a la que puede estar un cliente de la parada del camión para ser visitado por un repartidor
a bici es de $dist\_max$.
\end{enumerate}
La solución debe indicar el recorrido del camión, especificando qué clientes visitados a pie/bici están asociados a cada una de las paradas que realiza. Considerar que:
\begin{enumerate}
    \item Que un cliente se encuentre a una distancia menor a $dist\_max$ de una parada, no significa que será atendido
por un repartidor a pie/bici. Podría ser una nueva parada del camión.
\item Por cuestiones de mantenimiento de los productos refrigerados, no puede haber más de una entrega de productos refrigerados a pie/bici por un mismo repartidor. Es decir, \textbf{no se puede enviar más de un producto refrigerado desde un mismo lugar de partida.\\} \end{enumerate}

{También queremos ver como afecta a los costos imponer las siguientes restricciones:}

\begin{enumerate}[resume]
\item Queremos asegurar que cada repartidor a pie/bici contratado realice al menos 4 entregas.
\item Que haya determinados clientes que deban ser visitados por el camión. $camion\_obligatorio_i$
\end{enumerate}
}

\section*{Modelo} 

\subsection{Variables}
\begin{itemize}
    \item $VC_{ij} \in \{1,0\}$. Indica si el camión viaja al cliente j desde el cliente i.
    \item $VB_{ij}\in \{1,0\}$. Indica si se viaja en bicicleta al cliente j desde el cliente i.
    \item \st{$cant\_bicis \in \mathbb{N}$. Indica la cantidad de repartidores a bicicleta que contratamos. } \textbf{Esta variable ya no la usamos, ahora usamos:\\}
    $\delta_i \in \{0,1\}.$ Representa si se hace algún envío en bicicleta desde i.
    %\item $\delta = \{0,1\} $. Indica si $dist_{ij} \leq dist\_max$ 
    \item $u_i \in \mathbb{R},\ \forall i \in [1, n]$. Variable auxiliar que representa el orden de visita del cliente $i$ en la ruta del camión. Se usa para evitar ciclos disjuntos en el recorrido.

\end{itemize}
\subsection{función objetivo:}
{Minimizar: $\sum_{i=1}^{n} \sum_{j=1}^{n} VC_{ij}*c_{ij} + VB_{ij}*costo\_repartidor$}
\subsection{sujeto a:}
\begin{itemize}
    \item $\sum_{i=1}^{n} VC_{ik} = \sum_{j=1}^{n} VC_{kj}, \forall k\in[1,n]$. Nos asegura la conservación de flujo del camión (sale de un cliente la misma cantidad de veces que entra). Esto implica que el recorrido del camión es un ciclo.
    \item $\sum_{j=1}^{n} VC_{kj} \leq 1, \forall k \in [0,n]$. Nos dice que el camión puede pasar como máximo 1 vez por cliente.
    \item $\sum_{j=1}^{n} VB_{kj} \leq n\sum_{i=1}^{n} VC_{ik}, \forall k \in [0,n]$. Nos dice que solo podemos hacer viajes en bicicleta desde un cliente, si a ese cliente llegamos en camión.
    \item $d_{ij} \leq dist\_max + d_{ij}(1-VB_{ij})$. Esto nos dice que si la distancia de i a j es mayor que la distancia máxima, no se puede ir en bici de i a j.
    \item $\sum_{i=1}^{n} VC_{ij} + VB_{ij} = 1, \forall j \in [1,n]$. Esto nos dice que a todos los clientes los visitamos una vez, y de una sola manera.
 \item $u_i - u_j + n \cdot VC_{ij} \leq n - 1 \quad \forall i \ne j,\ i,j \in [1, n]$. 
Esta restricción evita que haya ciclos disjuntos en el recorrido del camión. Se utiliza la formulación MTZ (Miller-Tucker-Zemlin), que ordena las visitas entre clientes para asegurar un solo tour.

    \item \st{$\sum_{i=1}^{n}\sum_{j=1}^{n} VB_{ij} * r_j \leq cant\_bicis $. Hay al menos 1 bici por cada producto refrigerado que se reparte desde cualquier punto.} \textbf{Esto está mal. corrección abajo}
    \item $\sum_{j=1}^{n} VB_{ij} * r_j \leq 1, \forall i \in [1,n]$. Esto nos dice que desde cada lugar de partida solo se puede entregar a lo sumo 1 producto refrigerado.\\
    Otra versión sin usar variable $r_j$:\\
    $\sum_{j \in refrigerados} VB_{ij} \leq 1, \forall i \in [1,n]$
\end{itemize}
\vspace{0.5cm}
{También modelamos las restricciónes extra que queremos estudiar:\\}
\begin{itemize}
    \item  $\sum_{i=1}^{n} VC_{ij} \geq camion\_obligatorio_j, \forall j \in [1,n] $. Algunos clientes son visitados por el camión obligatoriamente
    \item \st{$\sum_{i=1}^{n}\sum_{j=1}^{n} VB_{ij}/4 \leq cant\_bicis$. Cada bici hace al menos 4 viajes.} \textbf{También está mal, había interpretado los repartidores como que salían desde el camión, pero solo hay 1 bici/repartidor por cliente dijo la profe.}\\
    Entonces lo reescribimos de la siguiente manera:
    \begin{enumerate}
        \item $  \sum_{j=1}^{n} VB_{ij} \leq n*\delta_i, \forall i \in [1,n]$
        \item $\sum_{j=1}^{n} VB_{ij} \geq 4*\delta_i, \forall i \in [1,n] $
    \end{enumerate}
    Donde $\delta_i$ representa si se hace algún envío en bicicleta desde i. Entonces decimos que se deben repartir o más de 4 productos en bicicleta desde i, o ninguno.
    %\item $\sum_{i=1}^{n}\sum_{j=1}^{n} VC_{ij} + VB_{ij} = n$ Esta restricción nos dice que todos los clientes se visitan
\end{itemize}



\renewcommand{\refname}{Bibliografía}
\bibliographystyle{plain-abbr}
%admite tesis de licenciatura como @LICTHESIS
\bibliography{bibliography}

\end{document}